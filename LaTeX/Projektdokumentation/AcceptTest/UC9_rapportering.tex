\begin{longtable}{| l | >{\raggedright}X | >{\raggedright}X | >{\raggedright}X | >{\raggedright\arraybackslash}p{2.3cm} |} \hline
	\multicolumn{2}{|l|}{\textbf{Use case under test}} & \multicolumn{3}{l|}{UC9: "Rapporting"} \\ \hline
	\multicolumn{2}{|l|}{\textbf{Scenarie}} & \multicolumn{3}{l|}{Hovedscenarie} \\ \hline
	\multicolumn{2}{|l|}{\textbf{Forudsætning}} & \multicolumn{3}{p{10.2cm}|}{UC 10 er aktivt, systemet er operationelt og E-mail-opsætning er udført af brugeren. Desuden skal brugeren have angivet ønske om at modtage notifikationer. Jordfugtighedssensor 1 er konfigureret til en plante, som har niveau 10 som ønsket jordfugtighedsparameter.\hfill} \\ \hline
	%\multicolumn{5}{|l|}{}\\ \hline
	\textbf{Step} & \textbf{Handling} & \textbf{Forventet Resultat} & \textbf{Resultat} & \textbf{Godkendt / Kommentar} \\ \hline
    9.1 & Bruger tjekker sin email klokken 12:15. & Visuel test: Bruger har modtaget E-mail med dalig status fra systemet. & ~ & ~ \\ \hline
    9.2 & Bruger tager jordfugtighedssensor 1 op af jorden. & Visuel test: Bruger modtager E-mail på tilmeldte E-mail adresse. & ~ & ~ \\ \hline
	\caption{Accepttest for UC9: Rapporting}\label{tbl:acceptUC9}
\end{longtable}