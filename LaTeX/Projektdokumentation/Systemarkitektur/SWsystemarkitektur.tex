\section{Application model}

Applikations modellen er valgt udfra udviklerene synspunkt og bruges for at give overblik over hvilke klasser som skal laves, og hvilket ansvar de hver især har. Nedenstående UML skal ses som core-systemet  og menu klasserne er udeladet for at skabe overblik. 

\begin{figure}[!h]
\centering 
\includegraphics[scale=0.8] {../fig/UML_autogreen.pdf}
\caption{Application model for AutoGreen}
\label{fig:UML}
\end{figure}

\subsection{Controller Klasser}

\subsubsection{Devkit8000}

Devkit8000 klassen skal initiere systemet og har derfter ansvaret for styring af processerne regulering og monitoring.
Devkit8000 klassen indeholder alle menuer beskrevet i menu oversigt. Brugeren kan interagere med klassen igennem menuerne.
Controller klassen har igennem menu tilgang til de alle andre klasser i systemet.


\subsection{Boundary Klasser}

\subsubsection{Monitor}

Monitor klassens primære opgave er at request sensordata fra UART klassen og skrive dem til data-loggen. Der ud over skal Monitor skrive til System-log, hvis UART klassen rapporter fejl ved data overførelse.

\subsubsection{Regulator}

Regulerings klassen har ansvaret for at planterværdierne bliver overholdt. Den opnår dette ved at læse fra dataloggen og hvis uregelmæssigheder findes blandt disse data, vil klassen tænde de fornødende akutuatorer gennem UART klassen. Der ud over skal Regulator skrive til System-log, hvis UART klassen rapporter fejl ved data overførelse.

\subsubsection{UART}

UART klassen er grænsefladen mellem devkittet og de sensorere/akutuatorere, der måtte eksisterer i AutoGreen systemet.

\subsubsection{Rapport}

Rapportering indlæser E-mail konfigurationer fra indstillinger, som bestemmer hvilke funktionaliteter skal benyttes. Rapporting skal sende email til brugeren dagligt, når der er kritisk klima i drivhuset, eller både dagligt og ved kritisk klima. 

\subsection{Domain Klasser}

\subsubsection{Data-log}

Data-loggen styrer en datastruktur. Det er dens opgave at modtage og indsætte målte planteværdier i datastrukturen, samt hente informationer ud fra strukturen.

\subsubsection{System-log}

System-loggen har til ansvar at styrer en datastruktur med henblik på at gemme de vigtigste systemhændelser og skal kunne tilgåes af brugeren senere.

\subsubsection{Indstillinger}

Indstillinger gemmer konfigurationer og indlæser dem i konfigurationsfilen, når regulering eller rapportering startes af brugeren. 


\subsubsection{Plantedatabasen}

Plantedatabasen gemmer parametre for bruger definerede planter samt  prekonfigurede planter og tilgåes via en QT klasse menu.

\clearpage

\section{Menu Oversigt}

Menu oversigten giver et overblik over de forskellige menuer og hvilke menuer, der giver tilgang til hinanden.

\begin{figure}[!h]
\centering 
\includegraphics[scale=0.7] {../fig/menu_oversigt.pdf}
\caption{Oversigt over AutoGreen's menuer}
\label{fig:QTMenu}
\end{figure}

\subsection{Menu beskrivelse}
Menu oversigten er med til at give et overblik over hvordan de forskellige menuer tilgåes igennem systemet, og fra hvilke menuer man kan tilgå andre menuer. Hovedmenuen er som standard stedet, hvor brugeren starter, da er her muligt at monitorere drivhusklimaet. I hovedmenuen har brugeren mulighed for at tilgå de 5 undermenuer: virtuel drivhus-, historik-, plantedatabase-, systemlog- og konfigurationsmenu.
\subsubsection{Virtuelle drivhusmenu}
I det virtuelle drivhus har brugeren mulighed for at tilføje nye planter til drivhuset, redigere allerede tilstedeværende planter, og herunder slette planter fra drivhuset. Uanset ønsket skal brugeren tilgå planteredigeringsmenuen.
\subsubsection{Historikmenu}
I historikmenuen har brugeren mulighed for at se data over drivhuset op til et år tilbage.
\subsubsection{Plantedatabasemenu}
I plantedatabasemenuen har brugeren mulighed for at tilføje nye planter til databasen, ved tryk på 'tilføj plante' oprettes en ny tom virtuel plante i databasen. Denne virtuelle plante åbnes i databaseredigeringsmenuen, hvor dens paramentre kan indstilles efter behov. Hvis brugeren ønsker at redigere allerede oprettede planter eller slette disse, kan brugeren trykke på den ønskede plante. Den valgte plante vil blive åbnet gennem databaseredigeringsmenuen, og det er her muligt at redigere eller slette planten.
\subsubsection{Systemlogmenu}
I systemloggen har brugeren mulighed for at se systemhændelser, f.eks. hvis systemet vælger at åbne et vindue, starte en blæser, eller bruge varmelegemet.
\subsubsection{Konfigurationsmenu}
I konfigureationsmenu har brugeren mulighed for at tilgå 4 undermenuer: E-mailmenu, Notifikationsmenu, Tid- og datomenu, samt Hardware Indstillingsmenu.

\subsubsection{E-mailmenu}
I E-mail menuen, vises 3 kolonner, hvor brugeren har mulighed for at indtaste E-mail, som skal modtage notifiktationer. 
\subsubsection{Notifikationsmenu} 
I notifikationsmenuen har brugeren mulighed for at slå notifikationer til og fra for både advarsels-notifikationer og daglige notifikationer. 
\subsubsection{Tids- og datomenu} 
I Tids- og datomenuen har brugeren mulighed for at ændre dato og tid. 
\subsubsection{Hardware Indstillingsmenu} 
I Hardware Indstillingsmenu har brugeren mulighed for at vælge hvilke akutuatorer drivhuset skal bruge. Hvis brugeren ønsker at sparer strøm, kan blæser og varmelegeme fravælges til regulering temperaturen. \\
