\subsection{SystemLog}

Systemloggen anvendes til at gemme data omkring systemhændelser angående: ændringer i den virtuelle og normale plantedatabase, aktivering af aktuatorerne, samt sendte beskeder til brugeren. Systemloggen kan tilgås af brugeren i Systemlogmenuen.


\subsubsection{Attributter}

\begin{table}[h]
\begin{tabularx}{\textwidth}{| L{2.5cm} | l | Z |} \hline
\texttt{SystemMsg} & \texttt{DoublyLinkedList <string> } & SystemMsg implementeres med en DoublyLinkedList, som gemmer i alle systemhændelser i form af strings.  \\\hline
\end{tabularx}
\caption{Attributter for klassen SystemLog}
\label{table:SystemLog_attributter}
\end{table}

\subsubsection{Metoder}

\begin{table}[h]
\begin{tabularx}{\textwidth}{| L{2.5 cm} | Z |} \hline
Prototype & \texttt{void AddMessage(string msg)} \\\hline
Parametre & \texttt{msg} \newline
Er en besked indeholdende den pågældende systemhændelse formateret på følgende måde: ”klassenavn”: ”hændelse” på ”tidspunkt”. \\\hline
Returværdi & \texttt{-} \\\hline
Beskrivelse & Funktionen har til formål at modtage systembeskeder fra andre klasser og indsætte disse beskeder i en datastruktur til senere brug. \\\hline
\end{tabularx}
\caption{AddMessage}
\label{table:SystemLog_AddMessage}
\end{table}

\begin{table}[h]
\begin{tabularx}{\textwidth}{| L{2.5 cm} | Z |} \hline
Prototype & \texttt{void PrintSystemLog()} \\\hline
Parametre & \texttt{-} \\\hline
Returværdi & \texttt{-} \\\hline
Beskrivelse & PrintSystemLog bliver kaldt fra QT menuen ”Systemlogmenu” og udskriver de sidste 5 systemhændelser med den femte ældste hændelse først og den nyeste hændelse sidst. \\\hline
\end{tabularx}
\caption{PrintSystemLog}
\label{table:SystemLog_PrintSystemLog}
\end{table}