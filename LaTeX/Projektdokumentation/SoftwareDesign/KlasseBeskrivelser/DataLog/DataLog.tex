\subsection{Datalog}

Dataloggen er en klasse som anvendes til at gemme data omkring klimaet i drivhuset i. Dataloggen består af en linked DoublyLinkedList \ref{sec:DoublyLinkedList} , som indeholder data over hver parameter i drivhuset, der bliver logget hvert minut.

\subsubsection{Attributter}

\begin{table}[h]
\begin{tabularx}{\textwidth}{| L{2.5 cm} | l | Z |} \hline
\texttt{list} & \texttt{DoublyLinkedList- <SensorData>} & En DoublyLinkedList der bruges til at gemme gemme i SensorData. \\\hline
\end{tabularx}
\caption{Attributter for klassen Datalog}
\label{table:Datalog_attributter}
\end{table}

\subsubsection{Metoder}

\begin{table}[h]
\begin{tabularx}{\textwidth}{| L{2.5 cm} | Z |} \hline
Prototype & \texttt{void InsertSensorData(SensorData sensorData)} \\\hline
Parametre & \texttt{SensorData SensorData} \newline
Indeholder det sensordata som du gerne vil indsætte i dataloggen. \\\hline
Returværdi & \texttt{-} \\\hline
Beskrivelse & Bruges til at indsætte sensorData i dataloggen. \\\hline
\end{tabularx}
\caption{InsertSensorData}
\label{table:Datalog_InsertSensorData}
\end{table}


\begin{table}[h]
\begin{tabularx}{\textwidth}{| L{2.5 cm} | Z |} \hline
Prototype & \texttt{SensorData GetNewestData()} \\\hline
Parametre & \texttt{-}\\\hline
Returværdi & Det nyste SensorData som  ligger gemt indeholder. \\\hline
Beskrivelse & Metoden har til formål at hente de seneste indsatte data i linked listen og returnere dem. \\\hline
\end{tabularx}
\caption{GetNewestData}
\label{table:Datalog_GetNewestData}
\end{table}


\begin{table}[h]
\begin{tabularx}{\textwidth}{| L{2.5 cm} | Z |} \hline
Prototype & \texttt{void Sortday()} \\\hline
Parametre & \texttt{-}\\\hline
Returværdi & \texttt{-} \\\hline
Beskrivelse & Metoden går ind i link listen fra nyeste data og går tilbage indtil at tiden passer med 24 timer før den nuværende tid, herefter tages data, 24 timer længere tilbage, og regnes sammen til en gennemsnitlig temperatur, luftfugtighed, lysintensitet og for op til 6 jordfugtigheder. De data der udtages af link listen slettes, og en ny Node oprettes på den først udtages plads. \\\hline
\end{tabularx}
\caption{Sortday}
\label{table:Datalog_Sortday}
\end{table}

\clearpage

\begin{table}[h]
\begin{tabularx}{\textwidth}{| L{2.5 cm} | Z |} \hline
Prototype & \texttt{void Sortweek()} \\\hline
Parametre & \texttt{-}\\\hline
Returværdi & \texttt{-} \\\hline
Beskrivelse & Metoden går ind i link listen fra nyeste data og går tilbage indtil at tiden passer med 2 dage før den nuværende tid, herefter tages data, 7 dage længere tilbage, og regnes sammen til en gennemsnitlig temperatur, luftfugtighed, lysintensitet og for op til 6 jordfugtigheder. De data der udtages af link listen slettes, og en ny Node oprettes på den først udtages plads. \\\hline
\end{tabularx}
\caption{Sortweek}
\label{table:Datalog_Sortweek}
\end{table}


\begin{table}[ht]
\begin{tabularx}{\textwidth}{| L{2.5 cm} | Z |} \hline
Prototype & \texttt{void Sortmonth()} \\\hline
Parametre & \texttt{-}\\\hline
Returværdi & \texttt{-} \\\hline
Beskrivelse & Metoden går ind i link listen fra nyeste data og går tilbage indtil at tiden passer med 8 dage før den nuværende tid, herefter tages data, 30 dage længere tilbage, og regnes sammen til en gennemsnitlig temperatur, luftfugtighed, lysintensitet og for op til 6 jordfugtigheder. De data der udtages af link listen slettes, og en ny Node oprettes på den først udtages plads. \\\hline
\end{tabularx}
\caption{Sortmonth}
\label{table:Datalog_Sortmonth}
\end{table}

\clearpage