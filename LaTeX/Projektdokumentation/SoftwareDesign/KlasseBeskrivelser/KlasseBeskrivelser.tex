\clearpage

\subsection{Klassebeskrivelser}

\subsection{Domainklasse Datalog}

\textbf{Attributter}

\begin{table}[h]
\begin{tabularx}{\textwidth}{| >{\raggedright\arraybackslash}X | >{\raggedright\arraybackslash}X | >{\raggedright\arraybackslash}p{10 cm} |} \hline
\texttt{DatalogList} & \texttt{datalog} & datalogList er en nedarvning af doublylinkedlist, med extra variabler til lagring af tid, temperatur, lysintensitet, luftfugtighed, og 6 jordfugtigheder. \\\hline
\end{tabularx}
\caption{Attributter for klassen Datalog}
\label{table:Datalog_attributter}
\end{table}

\textbf{Metoder}

\begin{table}[h]
\begin{tabularx}{\textwidth}{| >{\raggedright\arraybackslash}p{2.5 cm} | >{\raggedright\arraybackslash}X |} \hline
Prototype & \texttt{void GetData(int time, int temp, int light, int humidity, int ground)} \\\hline
Parametre & \texttt{int time} \newline 
time er en reference til et array som indeholder tidsstempler over en hvis periode. \newline
\texttt{int temp} \newline
temp er en reference til et array som indeholder temperaturen på de angivne tidstempler. \newline
\texttt{int light} \newline
light er en reference til et array som indeholder lysintensiteten på de angivne tidstempler. \newline
\texttt{int humidity} \newline
ground er en reference til et 2D array, hvor hver kolonne indeholde jordfugtigheden for den givne plante. Hver række indeholder jordfugtigheden på de angive tidstempler. \newline
\texttt{int ground} \newline
temp er en reference til et array som indeholder temperaturen på de angivne tidstempler. \\\hline
Returværdi & - \\\hline
Beskrivelse & Metoden har til fordel at hente data ud i et angivet tidsområde, ved at indsætte disse data ind i referencerne, hvorefter metoden afsluttes. \\\hline
\end{tabularx}
\caption{GetData}
\label{table:GetData}
\end{table}

\begin{table}[h]
\begin{tabularx}{\textwidth}{| >{\raggedright\arraybackslash}p{2.5 cm} | >{\raggedright\arraybackslash}X |} \hline
Prototype & \texttt{void InsertSensorData(const SensorData SensorData)} \\\hline
Parametre & \texttt{int time} \newline 
time er en reference til et array som indeholder tidsstempler over en hvis periode. \\\hline
Returværdi & - \\\hline
Beskrivelse & Når metoden kaldes oprettes en ny Node i den linked list hvor alt dataen fra parameteren SensorData bliver lagret i. \\\hline
\end{tabularx}
\caption{InsertSensorData}
\label{table:InsertSensorData}
\end{table}

\begin{table}[h]
\begin{tabularx}{\textwidth}{| >{\raggedright\arraybackslash}p{2.5 cm} | >{\raggedright\arraybackslash}X |} \hline
Prototype & \texttt{void GetNewestData(int temp, int humidity, int plant)} \\\hline
Parametre & \texttt{int temp} \newline 
temp er en reference til en int, som indeholder den temperatur fra den nyeste node i linked listen. \newline
\texttt{int humidity} \newline
humidity er en reference til en int, som indeholder luftfugtigheden fra den nyeste node i linked listen. \newline
\texttt{int plant} \newline
ground er en reference til et int array, som indeholde jordfugtigheden for planterne fra den nyeste node i linked listen \\\hline
Returværdi & - \\\hline
Beskrivelse & Metoden går ind i link listen fra nyeste data og går tilbage indtil at tiden passer med 24 timer før den nuværende tid, herefter tages data, 24 timer længere tilbage, og regnes sammen til en gennemsnitlig temperatur, luftfugtighed, lysintensitet og for op til 6 jordfugtigheder. De data der udtages af link listen slettes, og en ny Node oprettes på den først udtages plads. \\\hline
\end{tabularx}
\caption{GetNewestData}
\label{table:GetNewestData}
\end{table}


\begin{table}[h]
\begin{tabularx}{\textwidth}{| >{\raggedright\arraybackslash}p{2.5 cm} | >{\raggedright\arraybackslash}X |} \hline
Prototype & \texttt{void Sortweek()} \\\hline
Parametre & - \\\hline
Returværdi & - \\\hline
Beskrivelse & Metoden går ind i link listen fra nyeste data og går tilbage indtil at tiden passer med 2 dage før den nuværende tid, herefter tages data, 7 dage længere tilbage, og regnes sammen til en gennemsnitlig temperatur, luftfugtighed, lysintensitet og for op til 6 jordfugtigheder. De data der udtages af link listen slettes, og en ny Node oprettes på den først udtages plads. \\\hline
\end{tabularx}
\caption{Sortweek}
\label{table:Sortweek}
\end{table}

\begin{table}[h]
\begin{tabularx}{\textwidth}{| >{\raggedright\arraybackslash}p{2.5 cm} | >{\raggedright\arraybackslash}X |} \hline
Prototype & \texttt{void Sortmonth()} \\\hline
Parametre & - \\\hline
Returværdi & - \\\hline
Beskrivelse & Metoden går ind i link listen fra nyeste data og går tilbage indtil at tiden passer med 8 dage før den nuværende tid, herefter tages data, 30 dage længere tilbage, og regnes sammen til en gennemsnitlig temperatur, luftfugtighed, lysintensitet og for op til 6 jordfugtigheder. De data der udtages af link listen slettes, og en ny Node oprettes på den først udtages plads. \\\hline
\end{tabularx}
\caption{Sortmonth}
\label{table:Sortmonth}
\end{table}


\clearpage

\subsection{Domainklasse Indstillinger}

\textbf{Attributter}

\begin{table}[h]
\begin{tabularx}{\textwidth}{| >{\raggedright\arraybackslash}X | >{\raggedright\arraybackslash}X | >{\raggedright\arraybackslash}p{10 cm} |} \hline
\texttt{virtuallePlants} & \texttt{Plant} & Indeholder 6 structs af typen plant \\\hline
\texttt{Email} & \texttt{String} & Indeholder 3 e-mails \\\hline
\texttt{Warning} & \texttt{bool} & En bool som fortæller om advarsels e-mails er slået til eller fra.\\\hline
\texttt{daily} & \texttt{bool} & En bool som fortæller om daglige e-mails er slået til eller fra. \\\hline
\texttt{Varmelegeme} & \texttt{bool} & En bool som fortæller om Varmelegemet er slået til eller fra.\\\hline
\texttt{blæserne} & \texttt{bool} & En bool som fortæller om blæserne er slået til eller fra. \\\hline
\texttt{monitor} & \texttt{bool} & En bool som fortæller om monitor er slået til eller fra. \\\hline
\texttt{regulering} & \texttt{bool} & En bool som fortæller om regulering er slået til eller fra. \\\hline
\end{tabularx}
\caption{Attributter for klassen Indstillinger}
\label{table:Indstillinger_attributter}
\end{table}

\textbf{Metoder}

\begin{table}[h]
\begin{tabularx}{\textwidth}{| >{\raggedright\arraybackslash}p{2.5 cm} | >{\raggedright\arraybackslash}X |} \hline
Prototype & \texttt{void SetMonitorering(bool active)} \\\hline
Parametre & \texttt{bool active} \newline
True hvis den skal kører ellers false. \\\hline
Returværdi & - \\\hline
Beskrivelse & Anvendes til sætte monitorering til at kører eller ikke kører. \\hline
\end{tabularx}
\caption{SetMonitorering}
\label{table:SetMonitorering}
\end{table}

\clearpage

\subsection{Boundaryklasse Monitor}

\textbf{Attributter}

\begin{table}[h]
\begin{tabularx}{\textwidth}{| >{\raggedright\arraybackslash}X | >{\raggedright\arraybackslash}X | >{\raggedright\arraybackslash}p{10 cm} |} \hline
\texttt{CurrentTime} & \texttt{Date} & Indeholder sidste tidpunkt modtaget fra Indstillinger. \\\hline
\texttt{Email} & \texttt{Notification} & Indeholder status for rapportering. \\\hline
\texttt{Virtuel} & \texttt{Plant} & Indeholder array af virtuelle planteparametre. \\\hline
\texttt{Real} & \texttt{Plant} & Indeholder array af faktiske planteparametre. \\\hline
\end{tabularx}
\caption{Attributter for klassen Monitor}
\label{table:Monitor_attributter}
\end{table}

\textbf{Metoder}

\begin{table}[h]
\begin{tabularx}{\textwidth}{| >{\raggedright\arraybackslash}p{2.5 cm} | >{\raggedright\arraybackslash}X |} \hline
Prototype & \texttt{void CompareData()} \\\hline
Parametre & - \\\hline
Returværdi & - \\\hline
Beskrivelse & CompareDatas opgave er en live opdatering af plante sundhedstegn på baggrund af en sammenligningen mellem hvad brugeren har angivet som de ideelle værdier i den virtuelle plantedatabase og de faktiske værdier indsamlet fra sensorer. Hvis de faktiske værdier ligger inden for tolerancen vil Monitor farve plantefelterne i hovedmenuen grønne og hvis de ligger udenfor skal plantefelterne være røde. I tilfælde af en sensor ikke er tilkoblet, så skal plantefeltet for den pågældende sensor være grå.Hver gang en tolerance overskides skal Monitor angive hændelsen til systemloggen og signaler rapportering for aktivering. \\\hline
\end{tabularx}
\caption{CompareData}
\label{table:CompareData}
\end{table}

\clearpage

\subsection{Domainklasse Plantedatabase}

\textbf{Attributter}

\begin{table}[h]
\begin{tabularx}{\textwidth}{| >{\raggedright\arraybackslash}X | >{\raggedright\arraybackslash}X | >{\raggedright\arraybackslash}p{10 cm} |} \hline
\texttt{dataBase} & \texttt{DoublyLinkedList} & Datastrukturen som indeholder alle planterne \\\hline
\end{tabularx}
\caption{Attributter for klassen Plantedatabase}
\label{table:Plantedatabase_attributter}
\end{table}

\textbf{Metoder}

\begin{table}[h]
\begin{tabularx}{\textwidth}{| >{\raggedright\arraybackslash}p{2.5 cm} | >{\raggedright\arraybackslash}X |} \hline
Prototype & \texttt{Plant GetPlantList()} \\\hline
Parametre & - \\\hline
Returværdi & \texttt{Plant} \newline
Et Plant array med alle planter I database. \\\hline
Beskrivelse & Bruges til at hente all planter i databasen. \\\hline
\end{tabularx}
\caption{GetPlantList}
\label{table:GetPlantList}
\end{table}

\begin{table}[h]
\begin{tabularx}{\textwidth}{| >{\raggedright\arraybackslash}p{2.5 cm} | >{\raggedright\arraybackslash}X |} \hline
Prototype & \texttt{Plant GetPlant(int id)} \\\hline
Parametre & \texttt{int id} \newline
id nummeret for den ønskede plante. \\\hline
Returværdi & \texttt{Plant} \newline
En plante som har det givne id. \\\hline
Beskrivelse & Henter en plant ud af databasen som har det angivne id. \\\hline
\end{tabularx}
\caption{GetPlant}
\label{table:GetPlant}
\end{table}

\begin{table}[h]
\begin{tabularx}{\textwidth}{| >{\raggedright\arraybackslash}p{2.5 cm} | >{\raggedright\arraybackslash}X |} \hline
Prototype & \texttt{void InsertPlant(Plant plant)} \\\hline
Parametre & \texttt{Plant plant} \newline
Den plante som skal indsættes. \\\hline
Returværdi & - \\\hline
Beskrivelse & Indsætter en plante i databasen. \\\hline
\end{tabularx}
\caption{InsertPlant}
\label{table:InsertPlant}
\end{table}

\begin{table}[h]
\begin{tabularx}{\textwidth}{| >{\raggedright\arraybackslash}p{2.5 cm} | >{\raggedright\arraybackslash}X |} \hline
Prototype & \texttt{void DeletePlant(int id)} \\\hline
Parametre & \texttt{int id} \newline
id på planeten som skal slettes \\\hline
Returværdi & - \\\hline
Beskrivelse & Sletter en planet som har det angivende id. \\\hline
\end{tabularx}
\caption{DeletePlant}
\label{table:DeletePlant}
\end{table}

\begin{table}[h]
\begin{tabularx}{\textwidth}{| >{\raggedright\arraybackslash}p{2.5 cm} | >{\raggedright\arraybackslash}X |} \hline
Prototype & \texttt{plant CreatePlant()} \\\hline
Parametre & - \\\hline
Returværdi & \texttt{Plant} \newline
En ny plante. \\\hline
Beskrivelse & Opretter en ny plante i Datastrukturen og returnere denne plante, så det er muligt at redigere i data på den ved brug af plantedataredigeringsmenuen. \\\hline
\end{tabularx}
\caption{CreatePlant}
\label{table:CreatePlant}
\end{table}

\clearpage

\subsection{Boundaryklasse Rapport}

\textbf{Attributter}

\begin{table}[h]
\begin{tabularx}{\textwidth}{| >{\raggedright\arraybackslash}X | >{\raggedright\arraybackslash}X | >{\raggedright\arraybackslash}p{10 cm} |} \hline
\texttt{CurrentTime} & \texttt{Date} & Indeholder sidste tidpunkt modtaget fra Indstillinger. \\\hline
\texttt{Email} & \texttt{Notification} & Indeholder maillingsliste for status emails \\\hline
\end{tabularx}
\caption{Attributter for klassen Rapport}
\label{table:Rapport_attributter}
\end{table}

\textbf{Metoder}

\begin{table}[h]
\begin{tabularx}{\textwidth}{| >{\raggedright\arraybackslash}p{2.5 cm} | >{\raggedright\arraybackslash}X |} \hline
Prototype & \texttt{void ActivateRapport()} \\\hline
Parametre & - \\\hline
Returværdi & - \\\hline
Beskrivelse & Sendes fra Monitor og angiver om rapporterings funktionaliteter skal bruges eller ej. \\\hline
\end{tabularx}
\caption{ActivateRapport}
\label{table:ActivateRapport}
\end{table}

\begin{table}[h]
\begin{tabularx}{\textwidth}{| >{\raggedright\arraybackslash}p{2.5 cm} | >{\raggedright\arraybackslash}X |} \hline
Prototype & \texttt{void SendDailyRapport()} \\\hline
Parametre & - \\\hline
Returværdi & - \\\hline
Beskrivelse & Sender en E-mail til brugeren med en liste over de seneste systemhændelser siden sidste E-mail. \\\hline
\end{tabularx}
\caption{SendDailyRapport}
\label{table:SendDailyRapport}
\end{table}

\begin{table}[h]
\begin{tabularx}{\textwidth}{| >{\raggedright\arraybackslash}p{2.5 cm} | >{\raggedright\arraybackslash}X |} \hline
Prototype & \texttt{void SendWarningRapport()} \\\hline
Parametre & - \\\hline
Returværdi & - \\\hline
Beskrivelse &  Sender en E-mail til brugeren med den kritiske systemhændelse. \\\hline
\end{tabularx}
\caption{SendWarningRapport}
\label{table:SendWarningRapport}
\end{table}

\clearpage

\subsection{Boundaryklasse Regulator}

\textbf{Attributter}

\begin{table}[h]
\begin{tabularx}{\textwidth}{| >{\raggedright\arraybackslash}X | >{\raggedright\arraybackslash}X | >{\raggedright\arraybackslash}p{10 cm} |} \hline
\texttt{TempHigh} & \texttt{bool} & som standard står TempHigh som False, men hvis temperature bliver for høj sættes den True. \\\hline
\texttt{TempLow} & \texttt{bool} & som standard står TempLow som False, men hvis temperature bliver for lav sættes den True. \\\hline
\texttt{humidityLow} & \texttt{bool} & som standard står humidityLow som False, men hvis luftfugtigheden bliver for lav sættes den True. \\\hline
\texttt{humidityHigh} & \texttt{bool} & som standard står humidityHigh som False, men hvis luftfugtigheden bliver for høj sættes den True.\\\hline
\texttt{plante1water 1-6} & \texttt{bool} & Plante(1-6)water står som standard til False, men hvis jordfugtigheden bliver 2 niveauer for lavt sættes den til True.\\\hline
\end{tabularx}
\caption{Attributter for klassen Regulator}
\label{table:Regulator_attributter}
\end{table}

\textbf{Metoder}

\begin{table}[h]
\begin{tabularx}{\textwidth}{| >{\raggedright\arraybackslash}p{2.5 cm} | >{\raggedright\arraybackslash}X |} \hline
Prototype & \texttt{void ControlData()} \\\hline
Parametre & - \\\hline
Returværdi & - \\\hline
Beskrivelse & Metoden har til formål at sammenligne den faktiske temperatur i drivhuset med den ønskede temperatur. Det samme fortages med luftfugtighed. I forhold til jordfugtigheden tjekkes hver plante, og hvis jordfugtigheden er for lav, sættes boolean til at den gældende plante mangler vand. \\\hline
\end{tabularx}
\caption{ControlData}
\label{table:ControlData}
\end{table}

\clearpage

\subsection{Domainklasse Systemlog}

\textbf{Attributter}

\begin{table}[h]
\begin{tabularx}{\textwidth}{| >{\raggedright\arraybackslash}X | >{\raggedright\arraybackslash}X | >{\raggedright\arraybackslash}p{10 cm} |} \hline
\texttt{SystemMsg} & \texttt{DoublyLinkedList} & SystemMsg implementeres som en struct, der nedarver fra Node klassen, og udvider dem med en message item af typen string. \\\hline
\end{tabularx}
\caption{Attributter for klassen Systemlog}
\label{table:Systemlog_attributter}
\end{table}

\textbf{Metoder}

\begin{table}[h]
\begin{tabularx}{\textwidth}{| >{\raggedright\arraybackslash}p{2.5 cm} | >{\raggedright\arraybackslash}X |} \hline
Prototype & \texttt{void AddMessage(string msg)} \\\hline
Parametre & \texttt{string msg} \newline
string msg er en besked indeholdende den pågældende systemhændelse formateret på følgende måde: ”klassenavn”: ”hændelse” på ”tidspunkt”. \\\hline
Returværdi & - \\\hline
Beskrivelse & Funktionen har til formål at modtage systembeskeder fra andre klasser og indsætte disse beskeder i en datastruktur til senere brug. \\\hline
\end{tabularx}
\caption{AddMessage}
\label{table:AddMessage}
\end{table}

\begin{table}[h]
\begin{tabularx}{\textwidth}{| >{\raggedright\arraybackslash}p{2.5 cm} | >{\raggedright\arraybackslash}X |} \hline
Prototype & \texttt{void PrintSystemLog()} \\\hline
Parametre & - \\\hline
Returværdi & - \\\hline
Beskrivelse & PrintSystemLog bliver kaldt fra QT menuen ”Systemlogmenu” og udskriver de sidste 5 systemhændelser med den femte ældste hændelse først og den nyeste hændelse sidst. \\\hline
\end{tabularx}
\caption{PrintSystemLog}
\label{table:PrintSystemLog}
\end{table}

\clearpage

\subsection{Boundaryklasse UART}

\textbf{Attributter} %TODO

\begin{table}[h]
\begin{tabularx}{\textwidth}{| >{\raggedright\arraybackslash}X | >{\raggedright\arraybackslash}X | >{\raggedright\arraybackslash}p{10 cm} |} \hline
\texttt{SystemMsg} & \texttt{DoublyLinkedList} & SystemMsg implementeres som en struct, der nedarver fra Node klassen, og udvider dem med en message item af typen string. \\\hline
\end{tabularx}
\caption{Attributter for klassen UART}
\label{table:UART_attributter}
\end{table}

\textbf{Metoder}

\begin{table}[h]
\begin{tabularx}{\textwidth}{| >{\raggedright\arraybackslash}p{2.5 cm} | >{\raggedright\arraybackslash}X |} \hline
Prototype & \texttt{void ScanForSensors()} \\\hline
Parametre & - \\\hline
Returværdi & - \\\hline
Beskrivelse & metoden sender en besked til PSoC masteren og beder om antallet af tilsluttede sensorer til systemet. UARTen sender kommandoen (REF til UART protokol), og venter derefter på svar fra PSoC masteren. Hvis den får et gyldigt svar tilbage afsluttes metoden. Hvis svaret ikke er gyldigt vil metoden kontakte PSoC masteren en gang til for at få antallet af tilsluttede sensorer. Efter 4 fejlforsøg afsluttes metoden, og systemet sender en besked til bruger at tjekke systemet. \\\hline
\end{tabularx}
\caption{ScanForSensors}
\label{table:ScanForSensors}
\end{table}

\begin{table}[h]
\begin{tabularx}{\textwidth}{| >{\raggedright\arraybackslash}p{2.5 cm} | >{\raggedright\arraybackslash}X |} \hline
Prototype & \texttt{SensorData GetSensorData() } \\\hline
Parametre & - \\\hline
Returværdi &\texttt{SensorData} \newline
SensorData, som er en struct over temperatur, luftfugtighed, lysintensitet og jordfugtighed returneres med værdierne for målt temperatur, luftfugtighed, lysintensitet og jordfugtighed. \\\hline
Beskrivelse & metoden sender via UART protokollen en anmodning til PSoC masteren om at få temperaturen i drivhuset. Når korrekt data er modtaget fortsætter metoden til at indsamle data for luftfugtighed, lysintensitet og jordfugtighed for de 6 planter. Hvis en fejl i en af beskederne opstår, forsøges yderlige 3 forsøg for den gældte data, før den springer den gældende data over og fortsætter til næste data.  \\\hline
\end{tabularx}
\caption{GetSensorData}
\label{table:GetSensorData}
\end{table}

