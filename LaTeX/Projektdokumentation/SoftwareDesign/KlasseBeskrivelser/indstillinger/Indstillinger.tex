\subsection{Indstillinger}

\subsubsection{Attributter}

\begin{table}[ht]
\begin{tabularx}{\textwidth}{| Z | Z | L{9 cm} |} \hline
\texttt{virtuallePlants} & \texttt{Plant[6]} & Indeholder 6 structs af typen Plant \\\hline
\texttt{Email} & \texttt{String[3]} & Indeholder 3 e-mails \\\hline
\texttt{Warning} & \texttt{bool} & En bool som fortæller om advarsels e-mails er slået til eller fra.\\\hline
\texttt{daily} & \texttt{bool} & En bool som fortæller om daglige e-mails er slået til eller fra. \\\hline
\texttt{Varmelegeme} & \texttt{bool} & En bool som fortæller om Varmelegemet er slået til eller fra.\\\hline
\texttt{blæserne} & \texttt{bool} & En bool som fortæller om blæserne er slået til eller fra. \\\hline
\texttt{monitor} & \texttt{bool} & En bool som fortæller om monitor er slået til eller fra. \\\hline
\texttt{regulering} & \texttt{bool} & En bool som fortæller om regulering er slået til eller fra. \\\hline
\end{tabularx}
\caption{Attributter for klassen Indstillinger}
\label{table:Indstillinger_attributter}
\end{table}

\subsubsection{Metoder}

\begin{table}[ht]
\begin{tabularx}{\textwidth}{| L{2.5 cm} | Z |} \hline
Prototype & \texttt{string exec(const char* cmd)} \\\hline
Parametre & En string med den shell kommando der skal eksekveres på systemet. \\\hline
Returværdi & \texttt{String} \newline
En string som indeholder outputtet fra den kørte shell kommando \\\hline
Beskrivelse & Kan kørere kommandoer på et Linux system via en pipe til systemet. \\\hline
\end{tabularx}
\caption{exec}
\label{table:Indstillinger_exec}
\end{table}


\begin{table}[ht]
\begin{tabularx}{\textwidth}{| L{2.5 cm} | Z |} \hline
Prototype & \texttt{void SetMonitorering(bool active)} \\\hline
Parametre & \texttt{bool active} \newline
True hvis den skal kører ellers false. \\\hline
Returværdi & - \\\hline
Beskrivelse & Anvendes til sætte monitorering til at kører eller ikke kører. \\\hline
\end{tabularx}
\caption{SetMonitorering}
\label{table:Indstillinger_SetMonitorering}
\end{table}

\begin{table}[ht]
\begin{tabularx}{\textwidth}{| L{2.5 cm} | Z |} \hline
Prototype & \texttt{void SetRegulering(bool active)} \\\hline
Parametre & \texttt{bool active} \newline
True hvis den skal kører ellers false. \\\hline
Returværdi & - \\\hline
Beskrivelse & Anvendes til sætte regulering til at kører eller ikke kører. \\\hline
\end{tabularx}
\caption{SetRegulering}
\label{table:Indstillinger_SetRegulering}
\end{table}

\begin{table}[ht]
\begin{tabularx}{\textwidth}{| L{2.5 cm} | Z |} \hline
Prototype & \texttt{bool getRegulering()} \\\hline
Parametre & \texttt{-} \newline
 \\\hline
Returværdi & True hvis den skal kører ellers false. \\\hline
Beskrivelse & Anvendes til at tjekke status på regulering, så man kan se om den kører eller ej. \\\hline
\end{tabularx}
\caption{getRegulering}
\label{table:Indstillinger_getRegulering}
\end{table}

\begin{table}[ht]
\begin{tabularx}{\textwidth}{| L{2.5 cm} | Z |} \hline
Prototype & \texttt{bool getMonitorering()} \\\hline
Parametre & \texttt{-} \newline
 \\\hline
Returværdi & True hvis den skal kører ellers false. \\\hline
Beskrivelse & Anvendes til at tjekke status på monitorering, så man kan se om den kører eller ej. \\\hline
\end{tabularx}
\caption{getMonitorering}
\label{table:Indstillinger_getMonitorering}
\end{table}

\begin{table}[ht]
\begin{tabularx}{\textwidth}{| L{2.5 cm} | Z |} \hline
Prototype & \texttt{void SetVirtualPlant(int id, Plant plantToPlace)} \\\hline
Parametre & \texttt{int id} \newline et id som er mellem 1 og 6 som svare til hvor planeten er i det Virtual drivhus. \newline
\texttt{Plant plantToPlace} \newline Planten som skal sættes ind i det virtual drivhus. \\\hline
Returværdi & - \\\hline
Beskrivelse & Bruges til at indsætte en plante ind i det virtuelle drivhus. På den ønskede placering \\\hline
\end{tabularx}
\caption{SetVirtualPlant}
\label{table:Indstillinger_SetVirtualPlant}
\end{table}

\begin{table}[ht]
\begin{tabularx}{\textwidth}{| L{2.5 cm} | Z |} \hline
Prototype & \texttt{void DelVirtualPlant(int id)} \\\hline
Parametre & \texttt{int id} \newline et id som er mellem 1 og 6 som svare til hvor planeten er i det Virtual drivhus. \\\hline
Returværdi & - \\\hline
Beskrivelse & Anvendes til at slette en virtuel plante i det virtuel drivhus. \\\hline
\end{tabularx}
\caption{DelVirtualPlant}
\label{table:Indstillinger_DelVirtualPlant}
\end{table}

\begin{table}[ht]
\begin{tabularx}{\textwidth}{| L{2.5 cm} | Z |} \hline
Prototype & \texttt{void GetEmails(string \&mail1, string \&mail2, string \&mail3)} \\\hline
Parametre & \texttt{string \&mail1} \newline En reference til en string, son ændres til E-mail et som er gemt i system. \newline
\texttt{string \&mail2} \newline En reference til en string, son ændres til E-mail to som er gemt i system.
\newline
\texttt{string \&mail3} \newline En reference til en string, son ændres til E-mail tre som er gemt i system.
 \\\hline
Returværdi & - \\\hline
Beskrivelse & Bruges til at hente de tre e-mail adresser som systemet har gemt. \\\hline
\end{tabularx}
\caption{GetEmails}
\label{table:Indstillinger_GetEmails}
\end{table}

\clearpage

\begin{table}[ht]
\begin{tabularx}{\textwidth}{| L{2.5 cm} | Z |} \hline
Prototype & \texttt{void SetEmails(const string mail1, const string mail2, const string mail3)} \\\hline
Parametre & \texttt{string \&mail1} \newline E-mail et som der kan ønskes gemt. \newline
\texttt{string \&mail2} \newline E-mail to som der kan ønskes gemt.
\newline
\texttt{string \&mail3} \newline E-mail et som der kan ønskes gemt.
 \\\hline
Returværdi & - \\\hline
Beskrivelse & Bruges til at gemme tre e-mail adresser i systemet. \\\hline
\end{tabularx}
\caption{SetEmails}
\label{table:Indstillinger_SetEmails}
\end{table}

\begin{table}[ht]
\begin{tabularx}{\textwidth}{| L{2.5 cm} | Z |} \hline
Prototype & \texttt{void GetHardware(bool \&Varmelegeme, bool \&bloeserne)} \\\hline
Parametre & \texttt{bool \&Varmelegeme} \newline En reference til en bool, som ændres til den aktuelle status på Varmelegemet.
\newline
\texttt{bool \&bloeserne} \newline En reference til en bool, som ændres til den aktuelle status på blæserne.
 \\\hline
Returværdi & - \\\hline
Beskrivelse & Anvendes til at se om varmelegemet og/eller blæserne må bruges til at regulere. Hvis den er true må hardwaren bruges hvis false må det ikke bruges til at regulere med. \\\hline
\end{tabularx}
\caption{GetHardware}
\label{table:Indstillinger_GetHardware}
\end{table}

\begin{table}[ht]
\begin{tabularx}{\textwidth}{| L{2.5 cm} | Z |} \hline
Prototype & \texttt{void SetHardware(const bool Varmelegeme, const bool bloeserne)} \\\hline
Parametre & \texttt{const bool Varmelegeme} \newline Sættes true hvis varmelegemet skal kører ellers false
\newline
\texttt{const bool bloeserne} \newline Sættes true hvis blæserne skal kører ellers false
 \\\hline
Returværdi & - \\\hline
Beskrivelse & Anvendes at sætte om Varmelegemet og/eller blæserne skal må bruges til regulering i drivhuset. \\\hline
\end{tabularx}
\caption{SetHardware}
\label{table:Indstillinger_SetHardware}
\end{table}

\clearpage

\begin{table}[ht]
\begin{tabularx}{\textwidth}{| L{2.5 cm} | Z |} \hline
Prototype & \texttt{void setDate(Date time)} \\\hline
Parametre & \texttt{Date time} \newline En struct af type Date som beskriver den tid som systemet skal indstilles til. \\\hline
Returværdi & - \\\hline
Beskrivelse & Anvendes at sætte tiden på systemet, dette sker ved at eksekvere et script bash script på systemet via en shell som smider de rette parameter ind. Systemet kan sættes til en tid mellem år 2000 og 2050. \\\hline
\end{tabularx}
\caption{setDate}
\label{table:Indstillinger_setDate}
\end{table}

\begin{table}[ht]
\begin{tabularx}{\textwidth}{| L{2.5 cm} | Z |} \hline
Prototype & \texttt{Date getDate()} \\\hline
Parametre & \texttt{-} \newline \\\hline
Returværdi & En struct af type Date som beskriver den aktuelle tid som på systemet. \\\hline
Beskrivelse & Anvendes henter den aktuelle tid på systemet.\\\hline
\end{tabularx}
\caption{getDate}
\label{table:Indstillinger_getDate}
\end{table}

\begin{table}[ht]
\begin{tabularx}{\textwidth}{| L{2.5 cm} | Z |} \hline
Prototype & \texttt{void GetNotifications(bool \&daily, bool \&Warning)} \\\hline
Parametre & \texttt{bool \&daily} \newline En reference til en bool, som ændres til den aktuelle status på daily notifications, hvis den ændres til true er daily slået til, men hvis den er false er daily ikke slået til.
\newline
\texttt{bool \&Warning} \newline En reference til en bool, som ændres til den aktuelle status på warning notifications, hvis den ændres til true er warning slået til, men hvis den er false er warning ikke slået til.
 \\\hline
Returværdi & - \\\hline
Beskrivelse & Anvendes til at se om daily og/eller Warning er slået til, hvis værdien for er true sendes notifications for den, men hvis false sendes den ikke.\\\hline
\end{tabularx}
\caption{GetNotifications}
\label{table:Indstillinger_GetNotifications}
\end{table}

\clearpage

\begin{table}[ht]
\begin{tabularx}{\textwidth}{| L{2.5 cm} | Z |} \hline
Prototype & \texttt{void SetNotifications(const bool \&daily, const bool \&Warning)} \\\hline
Parametre & \texttt{bool \&daily} \newline En reference til en bool, som bruges til at ændres den aktuelle status på daily notifications, hvis den ændres til true er daily slået til, men hvis ændres til false er daily ikke slået til.
\newline
\texttt{bool \&Warning} \newline En reference til en bool, som bruges til at ændres den aktuelle status på Warning notifications, hvis den ændres til true er Warning slået til, men hvis ændres til false er Warning ikke slået til.
 \\\hline
Returværdi & - \\\hline
Beskrivelse & Anvendes til gemme daily og/eller Warning om er slået til, hvis værdien for er true sendes notifications for den angivende notification, men hvis false sendes den/de ikke.\\\hline
\end{tabularx}
\caption{SetNotifications}
\label{table:Indstillinger_SetNotifications}
\end{table}

\begin{table}[ht]
\begin{tabularx}{\textwidth}{| L{2.5 cm} | Z |} \hline
Prototype & \texttt{Plant Getplant(int plantId)} \\\hline
Parametre & \texttt{int plantId} \newline Et id som beskriver hvilken plante der skal hentes ud. Id'et skal være et tal mellem 1 og 6.
\newline
 \\\hline
Returværdi & Plant \\\hline
Beskrivelse & Anvendes hente en plante ud af det virtuelle drivhus.\\\hline
\end{tabularx}
\caption{Getplant}
\label{table:Indstillinger_Getplant}
\end{table}

\begin{table}[ht]
\begin{tabularx}{\textwidth}{| L{2.5 cm} | Z |} \hline
Prototype & \texttt{plant* GetAll()} \\\hline
Parametre & \texttt{-} \newline

 \\\hline
Returværdi & Et array som indeholder de 6 virtuelle planter \\\hline
Beskrivelse & Anvendes hente en plante ud af systemet .\\\hline
\end{tabularx}
\caption{GetAll}
\label{table:Indstillinger_GetAll}
\end{table}
