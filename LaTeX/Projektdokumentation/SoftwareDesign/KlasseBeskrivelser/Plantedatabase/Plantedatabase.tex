\subsection{Plantedatabase}

Plantedatabasen er en klasse som anvendes til at gemme planter i. Den er indeholder en Doublylinkedlist. Ud over dette tilbyder den metoder så man kan interagere med den.


\subsubsection{Attributter}

\begin{table}[h]
\begin{tabularx}{\textwidth}{| L{2.5cm} | l | Z |} \hline
\texttt{dataBase} & \texttt{DoublyLinkedList} & Datastrukturen som indeholder alle planterne\\\hline
\end{tabularx}
\caption{Attributter for klassen Plantedatabase}
\label{table:Plantedatabase_attributter}
\end{table}

\subsubsection{Metoder}

\begin{table}[h]
\begin{tabularx}{\textwidth}{| L{2.5 cm} | Z |} \hline
Prototype & \texttt{Plant* GetPlantList()} \\\hline
Parametre & \texttt{-} \\\hline
Returværdi & Et Plant array med alle planter I database. \\\hline
Beskrivelse & Bruges til at hente all planter i databasen. \\\hline
\end{tabularx}
\caption{GetPlantList}
\label{table:Plantedatabase_GetPlantList}
\end{table}

\begin{table}[h]
\begin{tabularx}{\textwidth}{| L{2.5 cm} | Z |} \hline
Prototype & \texttt{Plant GetPlant(int id)} \\\hline
Parametre & \texttt{id} \newline
Er id nummeret for den ønskede plante. \\\hline
Returværdi & En plante som har det givne id. \\\hline
Beskrivelse & Henter en plant ud af databasen som har det angivne id. \\\hline
\end{tabularx}
\caption{GetPlant}
\label{table:Plantedatabase_GetPlant}
\end{table}

\begin{table}[h]
\begin{tabularx}{\textwidth}{| L{2.5 cm} | Z |} \hline
Prototype & \texttt{void InsertPlant(plant)} \\\hline
Parametre & \texttt{plant} \newline
Den plante som skal indsættes. \\\hline
Returværdi & \texttt{-} \\\hline
Beskrivelse & Indsætter en plante i databasen. \\\hline
\end{tabularx}
\caption{InsertPlant}
\label{table:Plantedatabase_InsertPlant}
\end{table}

\clearpage

\begin{table}[h]
\begin{tabularx}{\textwidth}{| L{2.5 cm} | Z |} \hline
Prototype & \texttt{void DeletePlant(int id)} \\\hline
Parametre & \texttt{id} \newline
id på planeten som skal slettes. \\\hline
Returværdi & \texttt{-} \\\hline
Beskrivelse & Sletter en planet som har det angivende id. \\\hline
\end{tabularx}
\caption{DeletePlant}
\label{table:Plantedatabase_DeletePlant}
\end{table}

\begin{table}[h]
\begin{tabularx}{\textwidth}{| L{2.5 cm} | Z |} \hline
Prototype & \texttt{plant CreatePlant()} \\\hline
Parametre & \texttt{-} \\\hline
Returværdi & En ny plante. \\\hline
Beskrivelse & Opretter en ny plante i Datastrukturen og returnere denne plante, så det er muligt at redigere i data på den ved brug af plantedataredigeringsmenuen. \\\hline
\end{tabularx}
\caption{CreatePlant}
\label{table:Plantedatabase_CreatePlant}
\end{table}