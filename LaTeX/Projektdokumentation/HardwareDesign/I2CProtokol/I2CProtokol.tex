\section{\IIC Protokol} \label{sec:I2C_protokol}
Dette afsnit omhandler kommunikationen mellem alle \IIC kommunikerende komponenter i projektet.

\subsection{Temperatursensor}
Temperatursensoren er valgt til at være LM75 \cite{lib:LM75}. Informationer til protokollen er fundet i databladet.

%Slave Temperatursensor
\begin{table}[h]
\centering
\begin{tabularx}{0.6\textwidth}{| X | X |} 			\hline
\textbf{Slave:} 		& Temperatursensor			\\ \hline
\textbf{Adresse:}		& 0x48						\\ \hline
\textbf{Bemærkninger:}	& scl: min 400kHz			\\ \hline
\end{tabularx}
\caption{\IIC Oplysninger for temperatursensor}
\label{tbl:I2CTempLuftOplysninger}
\end{table}

Når der skal læses data fra sensoren, sker det jf. Figur \ref{fig:I2CTempProtokol}, fundet i det respektive datablad \cite{lib:LM75}.

\begin{figure}[h]
\centering
\includegraphics[width=\textwidth - 1 cm]{../fig/LM75Bitsekvens.png}
\caption{\IIC read protokol for temperatursensor}
\label{fig:I2CTempProtokol}
\end{figure}

\clearpage

\subsection{Slave Aktuator}
Formålet med Slave Aktuatoren er at udføre forskellige opgaver, via et varmelegeme, blæsere, en motor og seks vandings bools. Disse opgaver bliver anmodet af Master PSoC. Under dette afsnit vil protokolen mellem Slave Aktuator og Master PSoC blive gennemgået.

%Slave Aktuator
\begin{table}[h]
\centering
\begin{tabularx}{0.6\textwidth}{| X | X |} 			\hline
\multicolumn{2}{ | l | }{\textbf{Adresse:} 0x42} 	\\ \hline
\textbf{Kommando} 		& \textbf{Beskrivelse}		\\ \hline
WriteAdjustWindow		& Åbning/Lukning af vindue	\\ \hline
WriteAdjustHeat			& Tænd/Sluk for varme		\\ \hline
WriteAdjustVentilation	& Juster ventilation		\\ \hline
WriteAdjustIrrigation	& Juster vanding			\\ \hline
ReadStatus				& Anmodning om status		\\ \hline
\end{tabularx}
\caption{\IIC Kommandoer for Slave Aktuator}
\label{tbl:I2CAktuatorKommandoer}
\end{table}



%WriteAdjustWindow
\begin{table}[h]
\centering
\begin{tabularx}{0.6\textwidth}{| >{\centering\arraybackslash}X | >{\centering\arraybackslash}X | >{\centering\arraybackslash}X | >{\centering\arraybackslash}X | >{\centering\arraybackslash}X | >{\centering\arraybackslash}X | >{\centering\arraybackslash}X | >{\centering\arraybackslash}X |}	\hline
W7 & W6 & W5 & W4 & W3 & W2 & W1 & W0				\\ \hline
\multicolumn{2}{ | l | }{0x0} 						&
\multicolumn{2}{  l | }{Don't Cares}				&
\multicolumn{4}{  l | }{Position for vindue,}
\\
\multicolumn{2}{ | l | }{} 							&
\multicolumn{2}{  l | }{}							&
\multicolumn{4}{  l | }{0x0 = lukket, 0xF = åben}
\\ \hline
\end{tabularx}
\caption{\IIC Kommando WriteAdjustWindow}
\label{tbl:I2CAktuatorKommandoWriteAdjustWindow}
\end{table}

For at gøre systemet opgraderbar er der valgt at klargøre fire bits til position af vindue. Systemet som det er skal kun have mulighed for enten at åbne eller lukke vinduet. Det er dog mulighed for at give vinduet flere positioner mellem fuld åben og lukket.

%WriteAdjustHeat
\begin{table}[h]
\centering
\begin{tabularx}{0.6\textwidth}{| >{\centering\arraybackslash}X | >{\centering\arraybackslash}X | >{\centering\arraybackslash}X | >{\centering\arraybackslash}X | >{\centering\arraybackslash}X | >{\centering\arraybackslash}X | >{\centering\arraybackslash}X | >{\centering\arraybackslash}X |}	\hline
H7 & H6 & H5 & H4 & H3 & H2 & H1 & H0				\\ \hline
\multicolumn{2}{ | l | }{0x1} 						&
\multicolumn{3}{  l | }{Don't Care}					&
\multicolumn{3}{  l | }{Tænd/Sluk}
\\
\multicolumn{2}{ | l | }{} 							&
\multicolumn{3}{  l | }{}							&
\multicolumn{3}{  l | }{varmelegeme,}
\\
\multicolumn{2}{ | l | }{} 							&
\multicolumn{3}{  l | }{}							&
\multicolumn{3}{  l | }{0x0 = off, 0x7 = on}
\\ \hline
\end{tabularx}
\caption{\IIC Kommando WriteAdjustHeat}
\label{tbl:I2CAktuatorKommandoWriteAdjustHeat}
\end{table}

På samme måde som vinduet har systemet kun mulighed for enten at tænde eller slukke for varmelegemet. Der er der klargjort tre bits for at gøre det muligt at opgradere denne funktionalitet til et PWM signal, således at dette kan reguleres til trin mellem tændt og slukket.

\clearpage

%WriteAdjustVentilation
\begin{table}[h]
\centering
\begin{tabularx}{0.6\textwidth}{| >{\centering\arraybackslash}X | >{\centering\arraybackslash}X | >{\centering\arraybackslash}X | >{\centering\arraybackslash}X | >{\centering\arraybackslash}X | >{\centering\arraybackslash}X | >{\centering\arraybackslash}X | >{\centering\arraybackslash}X |}	\hline
V7 & V6 & V5 & V4 & V3 & V2 & V1 & V0				\\ \hline
\multicolumn{2}{ | l | }{0x2} 						&
\multicolumn{3}{  l | }{Don't Care}					&
\multicolumn{3}{  l | }{Tænd/Sluk}
\\
\multicolumn{2}{ | l | }{} 							&
\multicolumn{3}{  l | }{}							&
\multicolumn{3}{  l | }{ventilation,}
\\
\multicolumn{2}{ | l | }{} 							&
\multicolumn{3}{  l | }{}							&
\multicolumn{3}{  l | }{0x0 = off, 0x7 = on}
\\ \hline
\end{tabularx}
\caption{\IIC Kommando WriteAdjustVentilation}
\label{tbl:I2CAktuatorKommandoWriteAdjustVentilation}
\end{table}

Til regulering af blæsere er der klargjort tre bits. Systemet skal blot kunne regulere blæserne mellem tændt og slukket. Dermed er der mulighed for at udvide funktionaliteten af reguleringen til trin mellem tændt og slukket.

%WriteAdjustIrrigation
\begin{table}[h]
\centering
\begin{tabularx}{0.6\textwidth}{| >{\centering\arraybackslash}X | >{\centering\arraybackslash}X | >{\centering\arraybackslash}X | >{\centering\arraybackslash}X | >{\centering\arraybackslash}X | >{\centering\arraybackslash}X | >{\centering\arraybackslash}X | >{\centering\arraybackslash}X |}	\hline
I7 & I6 & I5 & I4 & I3 & I2 & I1 & I0				\\ \hline
\multicolumn{2}{ | l | }{0x3} 						&
\multicolumn{6}{  l | }{Værdi for pins til vanding,}
\\
\multicolumn{2}{ | l | }{} 							&
\multicolumn{6}{  l | }{I5: nr. 6 – I0: nr. 1,}
\\
\multicolumn{2}{ | l | }{} 							&
\multicolumn{6}{  l | }{1 = on, 0 = off}
\\ \hline
\end{tabularx}
\caption{\IIC Kommando WriteAdjustIrrigation}
\label{tbl:I2CAktuatorKommandoWriteAdjustIrrigation}
\end{table}

Til regulering af de seks vandingsbools er der klargjort seks bits. Da disse er bools, altså tændt/slukket signaler, er der ikke nogen grund til at klargøre mere plads til fremtidig opgraderinger.

%ReadStatus
\begin{table}[!h]
\begin{tabularx}{\textwidth}{| >{\centering\arraybackslash}X | >{\centering\arraybackslash}X | >{\centering\arraybackslash}X | >{\centering\arraybackslash}X | >{\centering\arraybackslash}X | >{\centering\arraybackslash}X | >{\centering\arraybackslash}X | >{\centering\arraybackslash}X | >{\centering\arraybackslash}X | >{\centering\arraybackslash}X | >{\centering\arraybackslash}X | >{\centering\arraybackslash}X | >{\centering\arraybackslash}X | >{\centering\arraybackslash}X | >{\centering\arraybackslash}X | >{\centering\arraybackslash}X |}	\hline
W3 & W2 & W1 & W0 & H2 & H1 & H0 & V2 & V1 & V0 & I5 & I4 & I3 & I2 & I1 & I0				\\ \hline
\multicolumn{4}{ | l | }{Position for vindue,} 			&
\multicolumn{3}{  l | }{Status for}						&
\multicolumn{3}{  l | }{Status for}						&
\multicolumn{6}{  l | }{Status for pins til vanding,}
\\
\multicolumn{4}{ | l | }{0x0 = lukket,} 				&
\multicolumn{3}{  l | }{Varmelegeme,}					&
\multicolumn{3}{  l | }{ventilation,}					&
\multicolumn{6}{  l | }{I5: nr. 6 – I0: nr. 1,}	
\\
\multicolumn{4}{ | l | }{0xF = åben} 					&
\multicolumn{3}{  l | }{1 = on,}						&
\multicolumn{3}{  l | }{0x0 = off,}						&
\multicolumn{6}{  l | }{1 = on,}
\\
\multicolumn{4}{ | l | }{} 								&
\multicolumn{3}{  l | }{0 = off}						&
\multicolumn{3}{  l | }{0x7 = on}						&
\multicolumn{6}{  l | }{0 = off}					
\\ \hline
\end{tabularx}
\caption{\IIC Kommando ReadStatus}
\label{tbl:I2CAktuatorKommandoReadStatus}
\end{table}

ReadStatus gør det muligt for Master PSoC at spørge hvilken status de forskellige aktuatorer har. Dette passer overens med de klargjorte bits til de forskellige reguleringer for at simplificere processen.

\clearpage

\subsection{Slave Jordfugt}

Formålet med Slave Jordfugt er at måle og bearbejde værdier læst fra op til seks jordfugt sensorer. Disse data bliver anmodet af Master PSoC og herefter spurgt efter. Under dette afsnit vil protokolen mellem Slave Jordfugt og Master PSoC blive gennemgået.

%Slave Jordfugt
\begin{table}[h]
\centering
\begin{tabularx}{0.6\textwidth}{| X | X |} 				\hline
\multicolumn{2}{ | l | }{\textbf{Adresse:} 0x32} 		\\ \hline
\textbf{Kommando} 		& \textbf{Beskrivelse}			\\ \hline
WriteDesiredSensor		& Sensor der ønskes data fra	\\ \hline
ReadDesiredSensor		& Anmodning om sensor data		\\ \hline
\end{tabularx}
\caption{\IIC Kommandoer for Slave Jordfugt}
\label{tbl:I2CJordfugtKommandoer}
\end{table}



%WriteSensorNumber
\begin{table}[h]
\centering
\begin{tabularx}{0.6\textwidth}{| >{\centering\arraybackslash}X | >{\centering\arraybackslash}X | >{\centering\arraybackslash}X | >{\centering\arraybackslash}X | >{\centering\arraybackslash}X | >{\centering\arraybackslash}X | >{\centering\arraybackslash}X | >{\centering\arraybackslash}X |}	\hline
S7 & S6 & S5 & S4 & S3 & S2 & S1 & S0				\\ \hline
\multicolumn{5}{ | l | }{Don't Care} 				&
\multicolumn{3}{  l | }{Værdi for sensor}
\\
\multicolumn{5}{ | l | }{} 							&
\multicolumn{3}{  l | }{nummer der ønskes}
\\
\multicolumn{5}{ | l | }{} 							&
\multicolumn{3}{  l | }{læsning af, ved}
\\
\multicolumn{5}{ | l | }{} 							&
\multicolumn{3}{  l | }{næste ReadStatus;}
\\
\multicolumn{5}{ | l | }{} 							&
\multicolumn{3}{  l | }{0x0 = sensor 1,}
\\
\multicolumn{5}{ | l | }{} 							&
\multicolumn{3}{  l | }{0x5 = sensor 6}
\\ \hline
\end{tabularx}
\caption{\IIC Kommando WriteDesiredSensor}
\label{tbl:I2CJordfugtKommandoWriteDesiredSensor}
\end{table}

Master PSoC har via denne kommando mulighed for, at fortælle Slave Jordfugt hvilken sensor der ønskes data fra ved næste læsning. Da der er seks sensorer er der klargjort tre bits til værdi af sensor nummer.

%ReadStatus
\begin{table}[h]
\centering
\begin{tabularx}{0.6\textwidth}{| >{\centering\arraybackslash}X | >{\centering\arraybackslash}X | >{\centering\arraybackslash}X | >{\centering\arraybackslash}X | >{\centering\arraybackslash}X | >{\centering\arraybackslash}X | >{\centering\arraybackslash}X | >{\centering\arraybackslash}X |}	\hline
S7 & S6 & S5 & S4 & S3 & S2 & S1 & S0				\\ \hline
\multicolumn{1}{ | l | }{Sensor status,}			&
\multicolumn{7}{  l | }{Sensor værdi, 0x1 - 0x64 (1 - 100)}
\\
\multicolumn{1}{ | l | }{0x1 = deaktive,} 			&
\multicolumn{7}{  l | }{}
\\
\multicolumn{1}{ | l | }{0x0 = aktive} 				&
\multicolumn{7}{  l | }{}
\\ \hline
\end{tabularx}
\caption{\IIC Kommando ReadDesiredSensor}
\label{tbl:I2CJordfugtKommandoReadDesiredSensor}
\end{table}

Ved læsning vil Slave Jordfugt besvare Master PSoC med en værdi, mellem 1 og 100, fra den sensor Master PSoC'en har bedt om data fra via WriteDesiredSensor. I tilfælde af at der er opstået en fejl vil MSB være høj.

\clearpage