\chapter{Konklusion}
\label{ch:Konklusion}

Hvis man kun ser på afsnit \ref{P-ch:Accepttest} \nameref{P-ch:Accepttest} på side \pageref{P-ch:Accepttest} i projektdokumentationen, kan det umiddelbart se ud som de opstillede krav for projektet langt fra er opfyldt, men hvis man sammenholder accepttesten med MoSCoW prioriteringen i afsnit \ref{P-ch:Projektformulering} \nameref{P-ch:Projektformulering} på side \pageref{P-ch:Projektformulering} i projektdokumentationen, ses det, at de vigtigste krav er opfyldt. 

At de mange uopfyldte krav overhovedet blev formuleret i de tidlige faser af projektarbejdet, var med fuldt overlæg. 
Gruppen ønskede at arbejde med et stort og realistisk projekt, og ville hellere skære arbejdsopgaver væk undervejs end løbe tør for stof at arbejde med. 
Der er gennem projektarbejdet løbende lavet bagudgående rettelser i dokumentationen efterhånden som arbejdet skred frem, men gruppen har valgt at fastholde alle krav i projektformuleringen. 

En stor del af de krav som ikke er opfyldt, er delvist opfyldt. 
Det skyldes at gruppen som udgangspunkt har sat ambitionerne højt, og så har vi skåret ned undervejs. 
Der er fx blevet arbejdet med både lysintensitets- og luftfugtighedssensorer, men komplikationer med disse betød, at de blev droppet sidst i forløbet. 
Der er dog designet og implementeret SW for sensorerne, så der kræves ikke meget mere arbejde for at kravene vedrørende dette er opfyldt. 
Af andre eksempler kan nævnes sytemlog og plantedatabase, der er delvist implementerede, men af hensyn til tidsplanen ikke blev gjort fuldstændig færdige. 

Det implementerede systems største udviklingspotentiale ligger i UART kommunikationen mellem DevKit8000 og PSoC Mater, se \ref{ch:Resultater_og_diskussion} \nameref{ch:Resultater_og_diskussion} på side \pageref{ch:Resultater_og_diskussion} for nærmere diskussion af dette. 

I de sidste faser af projektarbejdet er der i gruppen internt blevet talt en del om, at det nærmest er ærgerligt at forløbet er slut. 
Der er mange funktionaliteter som kunne færdigimplementeres og/eller optimeres, hvis der havde været mere tid til rådighed. 

\mbox{}

Gruppen har undervejs været meget tilfreds med projektarbejdets forløb; de erfaringer gruppen gjorde sig på sidste semester har gavnet forløbet, og gruppen har formået at udvikle sig yderligere. 
Der er ingen tvivl om at der er store fordele ved at arbejde sammen i en "gammel"\ gruppe, der ikke først skal til at lære hinanden at kende både socialt og fagligt. 

Gruppen er meget tilfreds med det realiserede system, men der er bred enighed om at gruppens største styrke ligger i planlægning og koordinering af arbejdet. 
Der blev - som på sidste semester - lagt en stram tidsplan fra start; der var lagt op til en periode på tre uger til skrivning af denne rapport. 
Tidsplanen kom - som forventet - til at skride undervejs, men gruppen som helhed har undervejs formået at have overblik over arbejdet og rette i tidsplanen og kravene for projektet. 
Derved har vi undgået at skulle lave makværk og lappeløsninger i slutningen af forløbet. 

\clearpage