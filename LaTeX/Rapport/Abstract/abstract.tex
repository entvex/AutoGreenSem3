\chapter{Abstract}
\label{ch:Abstract}

This documentation contains the development of a prototype, which is meant to be installed in a greenhouse. The system, AutoGreen, measures the temperature in the greenhouse and regulates it, by opening/closing windows, using ventilation and a heater. The system is also able to measure soil humidity, with its six soil humidity-sensors, and with the data, able to give the user a notice if it is time to water the plants.

During the first phases of the project, it was set to have air humidity sensors and light sensors in the greenhouse, logging and graphical preview of the logged data, a database with standard and customized plants, an e-mail system to inform the user of important event in the greenhouse, and much more. This was not implemented in the prototype.

During the development of the product, a miniature greenhouse was used. If AutoGreen was to be installed in a real greenhouse, the ventilators, heater and window motor should be scaled up to fit the size of the greenhouse.

AutoGreen’s user interface and controller are developed on an Embest Devkit8000 Evaluation board\cite{lib:DK8000}. The Devkit8000 communicates with a \IIC master, made on a PSoC 4 Pioneer Kit\cite{lib:psoc4_guide}, using UART communication. The master unit communicates with its two \IIC slaves. One of the slaves being the “Actuator” unit, and the other slave being the unit for measuring soil-humidity. The slaves are developed on a PSoC 4 Pioneer Kit. The temperature is measured using a LM75 with \IIC interface\cite{lib:LM75}.

The system can measure temperatures within the greenhouse with a precision of +/- 0.5 degrees. It is able to regulate the temperature with a precision of +/- 1 degree, up to 10 degrees above the surrounding temperature. 

In case of water shortage at one of the soil humidity sensors, a message is displayed on the Graphical interface, and a port on one of the \IIC slaves switches from low to high.

The soil humidity sensors adds an option to install an automatic watering system to the AutoGreen system. There are much more options to further develop the system. See chapter \ref{P-ch:Projektformulering} \nameref{P-ch:Projektformulering} (danish) on page \pageref{P-ch:Projektformulering} in the documentation for further information. 

The biggest obstacle in the product is the UART communication. A simplified version of UART is used, which means only Tx, Rx and ground reference is used, there are a few miscommunications between the UART and the \IIC master. The miscommunication never occurs when connecting a UART terminal directly to the \IIC master. See chapter \ref{P-ch:Accepttest} \nameref{P-ch:Accepttest} (danish) on page \pageref{P-ch:Accepttest} in the documentation for further information on results.

\clearpage