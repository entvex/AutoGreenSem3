\section{Fremtidigt Arbejde} \label{ch:Fremtidigt_arbejde}

Umiddelbart er de resterende krav, som ikke blev implementeret i projektforløbet, oplagte kandidater til fremtidigt arbejde.
Dette inkluderer måling af lysintensitet og luftfugtighed, mulighed for at tilføje planter i det virtuelle drivhus, interaktion med plantedatabase, grafer for samtlige måledata, e-mail notifikationer samt fuld systemlog.
For at e-mail notifikationer skal fungere ville det også være nødvendigt at implementere internetadgang på DevKit8000, hertil kan det tilføjes at mulighed for trådløst netværk ville være oplagt, da systemet skal sidde i brugerens hjem og ikke nødvendigvis være i nærheden af router el. lign., og man ville slippe for at trække ethernet kabel til sit drivhus.
Man kunne også overveje en eller anden form for trådløs kommunikation mellem DevKit8000 og PSoC 4 Master; derved undgår man at brugerfladen skal placeres i drivhuset. 

Udover disse opgaver er oplagte muligheder integration med et vandingssystem, så drivhuset er selvstyrende mht. vanding. 
Dette kunne gøres ved at åbne/lukke magnetventiler for hver plante i drivhuset. 
For at det ville fungere, bør der også implementeres en form for kalibrering af vandtilførslen i forhold til systemet, så en eventuel vandingsalgoritme ville kunne justere vandmængden fornuftigt.

Under projektforløbet har gruppen indset, at PSoC Master-blokkens funktionaliteter kunne integreres i DevKit8000 og herved spare en hel hardwareblok, da denne har mulighed for \IIC direkte på kittet.
Der er ligeleds også fordele i at lade den digitale filtrering foregå på DevKit8000, da denne har langt flere resourcer tilgængelige i forhold til digital filtrering i software end PSoC 4.
En anden mulighed ville være at lade den digitale signalbehandling foregå på PSoC 5, der også forefindes på PSoC 4 Pioneer Kit.
Denne er dog upraktisk at udvikle på, da den ikke tilbyder samme debugging funktionaliteter som PSoC 4 processoren.

Gruppen har under forløbet også luftet tanken om at implementere en mobilapp til monitorering og interaktion med systemet, denne ville fx kunne implementeres til Android og iOS.
En sådan app ville kunne give brugeren information og mulighed for interaktion med systemet når brugeren ikke er hjemme i en længere periode, som fx på ferie.
Brugeren ville ligeledes også kunne anvende appen udelukkende i stedet for touchskærmen på DevKit8000.

Med hensyn til selve hardwaren i drivhuset ville det være oplagt at integrere aktuator og jordfugtmåler på én PSoC 4 og placere denne på samme print som strømforsyning. 
Dette ville reducere prisen på produktet væsentligt samt skabe mere stabilitet ift. udefrakommende påvirkninger (elektromagnetisk stråling og mekanisk påvirkning).
Ligeledes ville det være en fordel at indpakke alle hardwareenheder i passende kabinetter.