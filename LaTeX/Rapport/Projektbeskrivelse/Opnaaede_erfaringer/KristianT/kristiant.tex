\subsection{Kristian T.}
Under dette projektforløb har jeg primært dannet mange erfaringer omkring projektopbygning og design på PSoC.
Vi er under projektforløbet stødt på mange små og nogle store problemer, som umiddelbart ikke var synlige ud fra et arktitektur- og designperspektiv.
Disse problemer har tvunget mig og gruppen ud i at lære mere omkring opbyggelsen af PSoC, og hvordan den er stykket sammen ift. fx interrupts, noget som vi ikke rigtig er kommet ind på i de øvrige kurser dette semester.
Jeg har også opnået en del erfaringer omkring fejlsøgning af en opstilling, og hvordan man undgår eventuelle faldgrupper, fx har vi haft dårlig erfaring med at sætte projektet sammen med enkelte harwin-stik, da det var besværligt at sætte sammen igen uden at have dokumentationen foran sig.

En anden væsentlig ting jeg føler, jeg har lært, er at samarbejde bedre med projektgruppen.
Dette er forhåbentlig en delt oplevelse, men jeg synes den er værd at tage med, da samarbejde på et højere plan også hjælper ved fremtidige projektgrupper.

