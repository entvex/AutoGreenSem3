\subsection{Kasper}
Dette semesterprojekt har været meget interessant, men har også været ret fustrerende. Starten af projektet, hvor jeg har arbjedet sammen med gruppen fra det tidligere semester var god, og det var tydeligt at gruppen havde udviklet sig i forhold til det samme forløb på sidste semester. Da gruppen delte sig op i hardware of software, var jeg en del af softwaregruppen. I forløbet med software-gruppen ville jeg gerne have lavet en del om. Vi troede vi var smarte at konstruere et grundsystem først, hvor vi lavede de fleste funktionaliteter, og derefter sammensætte det med QT, hvilken endte med at give os problemer, da vi kom til at skulle arbejde grundsystemet ind i QT. Dette gav dog viden til, at der i fremtiden skal tages hensyn til hvis jeg igen skal bygge et system op, og så ikke bygge et grundsystem op først, men ihvertfald finde ud af hvordan det skal bygges sammen med GUIen. Jeg har desuden fået en masse ud af at arbejde med tråde og QT i større omfang, da det har givet indsigt i hvordan, det er at arbejde med større systemer, hvor flere funktionaliteter skal køres samtidig, og der skal være en brugerflade på. Generalt har projektet været meget lærerigt at arbejde på.