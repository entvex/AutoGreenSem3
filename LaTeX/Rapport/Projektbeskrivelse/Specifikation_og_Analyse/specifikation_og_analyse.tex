\section{Specifikation og Analyse}
\label{ch:Specifikation_og_Analyse}
Dette afsnit omhandler specifikation og analyse i forbindelse med projektets opstillede krav, samt bearbejdelsen og tankerne bag udarbejdelsen af kravspecifikationen (se \ref{P-ch:Kravspec} \nameref{P-ch:Kravspec} side \pageref{P-ch:Kravspec} i projektdokumentationen).

\mbox{}

Der blev foretaget undersøgelser af de forskellige krav, der blev opstillet til projektet, for at sikre at kravene faktisk kunne opfyldes. Der blev efterfølgende diskuteret hvorvidt forskellige sensorer og aktuatorer skulle implementeres i systemet. Det resulterede i, at systemet skulle indeholde en temperatur-, jordfugt-, luftfugt- og en lysintensitetssensor.

Til styring af systemet blev det valgt at bruge DevKit8000 som embedded system, og PSoC 4 Pioneer Kits til at styre hardwaren med. Det blev bestemt at bruge en PSoC Master, som skulle have forbindelse til DevKit8000 igennem UART. UART blev valgt som kommunikationsvej, da der allerede var kendskab til UART. Valget af UART gjorde fejlfinding nemt, da det var muligt at teste ved brug af en PC. PC'en kunne læse, hvad der blev sendt, og skrive tilbage ved brug af tastaturet på PCen.
Til kommunikation mellem PSoCs blev det besluttet at bruge \IIC jf. projektoplægget.
Beslutningen om at buskommunikationen skulle være af typen \IIC blev taget på baggrund af flere forskellige faktorer. Modsat SPI er \IIC i stand til at sende data over relativt lange afstande. \IIC interfacet har desuden indbygget sikkerhed i standarden.

Til regulering af temperatur i drivhuset blev det valgt, at et vindue skulle kunne åbne og lukke, og fire ventilatorer skulle udskifte luften i drivhuset. Formålet med disse er at køle drivhuset ned. Der blev til opvarmning valgt en glødepære, som en simpel måde at opvarme drivhuset på.

Efter de fleste overordnede hardware beslutninger var blevet taget, blev der lavet en overordnet plan for den generelle funktionalitet og hvordan den grafiske brugerflade skulle se ud.

Der kunne efter de generelle beslutninger, skrives use case diagrammer over de ønskede processer og funktionaliteter, og derved give et godt overblik til at opstille endelige krav. Disse krav blev inddelt i funktionelle og ikke-funktionelle krav, således at de valgte arbejdsmodeller blev fulgt, og det var muligt at opstille en endelig kravspecifikation under processen. Dette endte ud i, at en accepttestspecifikation (se afsnit \ref{P-ch:Accepttest} \nameref{P-ch:Accepttest} side \pageref{P-ch:Accepttest} i projektdokumentationen) var mulig at udarbejde, dermed blev V-modellen fulgt.

\clearpage